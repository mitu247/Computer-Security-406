% need to include this in main.tex
% \usepackage{xcolor}
\section{Zeek Logs}

\large\textbf{conn.log}
\vspace{5pt}

\\
\normalsize
The \colorbox{gray!20}{conn.log} file generated by Zeek is a crucial log that records details about network connections, covering both stateful protocols like TCP and stateless protocols like UDP. This log captures essential "layer 3" and "layer 4" information about network communications, such as source and destination IP addresses and ports, protocol used (TCP/UDP), service identified (like HTTP or DNS), connection duration, data transferred, and the state of the connection. Each entry in the \colorbox{gray!20}{conn.log} provides a comprehensive overview of a single network connection, making it an invaluable resource for network analysis and security investigations. The \colorbox{gray!20}{conn.log} serves as a foundational element in Zeek's logging ecosystem, allowing analysts to track and correlate network activities effectively.
\\
\vspace{1cm}

\noindent \Large \textbf{dns.log}
\vspace{5pt}
\\
\normalsize
The \colorbox{gray!20}{dns.log} in Zeek is a detailed record of DNS (Domain Name System) queries and
responses observed in network traffic. This log tracks key DNS transaction details, including the source and destination IP addresses and ports, the query type (e.g., A for IPv4 addresses, AAAA for IPv6 addresses), the queried domain names, and the responses received, such as resolved IP addresses. It captures both successful queries and failed ones, providing insights into DNS activities, which are crucial for both normal network operations and identifying potential security issues like command-and-control communications or DNS-based data exfiltration. The \colorbox{gray!20}{dns.log} is invaluable for analysts looking to understand the DNS behavior of their network, troubleshoot issues, or detect malicious activities.
\\
\vspace{1cm}

\noindent \Large \textbf{http.log}
\vspace{5pt}
\\
\normalsize
Zeek's \colorbox{gray!20}{http.log} captures detailed information about HTTP traffic observed in the network. This includes data on both clear-text HTTP and decrypted HTTPS traffic, if HTTPS is made visible as HTTP through certain network configurations. Each entry in the \colorbox{gray!20}{http.log} provides comprehensive details about individual HTTP transactions, such as the source and destination IP addresses and ports, the HTTP method used (e.g., GET or POST), the requested URL, the HTTP version, user-agent string, the length of the request and response bodies, the status code returned by the server, and the MIME type of the response. In essence, the \colorbox{gray!20}{http.log} allows analysts to see who made a request, to whom, what was requested, and the response to that request. This log is crucial for understanding web-based activities on the network, identifying suspicious or malicious HTTP traffic, and troubleshooting web application issues. By correlating \colorbox{gray!20}{http.log} entries with other Zeek logs (like \colorbox{gray!20}{conn.log}), analysts can gain a more holistic view of network activities and potential security incidents.
\\
\vspace{1cm}


\noindent \Large \textbf{files.log}
\vspace{5pt}
\\
\normalsize
Zeek's \colorbox{gray!20}{files.log} records details about files observed in network traffic, leveraging Zeek's File Analysis framework. This log includes metadata such as source and destination IP addresses and ports, MIME types, and file sizes, but not the file content itself. To extract and write actual file content to disk, Zeek requires configuration to specify which file types to extract, such as executable files or documents.

The \colorbox{gray!20}{files.log} plays a key role in security analysis by tracking files transferred over protocols like HTTP or FTP. Analysts can use this log to identify potentially malicious file transfers, investigate file-based attacks, or simply monitor file activities for data loss prevention.

In practice, analysts would typically start with a conn.log entry to identify a connection of interest, use related entries in http.log or other protocol-specific logs to understand the context of the file transfer, and then consult \colorbox{gray!20}{files.log} for details about the transferred files. If configured to extract files, Zeek would save them to disk, allowing for further analysis such as hashing, scanning with antivirus tools, or manual inspection.

In summary, \colorbox{gray!20}{files.log} is essential for tracking and analyzing files observed in network traffic, providing a foundation for understanding file-based network activities and potential security threats.
\\
\vspace{1cm}



\noindent \Large \textbf{ftp.log}
\vspace{5pt}
\\
\normalsize
Zeek's \colorbox{gray!20}{ftp.log} provides a summary of FTP (File Transfer Protocol) activities, capturing details of the interactions between an FTP client and server. This log includes information such as the client's commands (e.g., login, file retrieval commands), the server's responses, and details about the data transfer channel (e.g., passive or active mode). The log entries typically include the user credentials used for the session, the commands issued, the server's replies, and any file transfers that occurred, including file paths and sizes.

FTP sessions involve two channels: a control channel for commands and responses (typically over port 21) and a data channel for actual file transfers. The data channel's port can vary, especially in passive mode where the server specifies a port for the client to connect to for data transfer.

The \colorbox{gray!20}{ftp.log} is particularly useful for tracking file transfers, identifying unauthorized access, or monitoring for data exfiltration attempts. Analysts can correlate entries in \colorbox{gray!20}{ftp.log} with conn.log entries using unique identifiers to get a comprehensive view of the session, including both command exchanges and data transfers.
\\
\vspace{1cm}


\noindent \Large \textbf{ssl.log}
\vspace{5pt}
\\
\normalsize
Zeek's \colorbox{gray!20}{ssl.log} records details of TLS (Transport Layer Security) encrypted traffic, which is not limited to HTTPS but includes various protocols secured by TLS, like SMTP. The log captures key information such as TLS version, cipher suite, server names (via the SNI field, unless encrypted with ESNI/ECH), certificate details, and more. This enables analysts to observe encrypted communications for signs of suspicious or malicious activity, despite the payload being encrypted.

With advancements like TLS 1.3, some traditional visibility into encrypted traffic, such as certificate details, is reduced. Additional techniques like JA3 and JA3S fingerprinting help by profiling TLS client and server configurations to identify potentially malicious or unusual traffic patterns, even when the session contents are encrypted. This aids in maintaining some level of insight into encrypted communications, which is crucial in modern network security monitoring.
\\
\vspace{1cm}



\noindent \Large \textbf{X509.log}
\vspace{5pt}
\\
\normalsize
Zeek's \colorbox{gray!20}{X509.log} captures detailed information about X.509 certificates observed during TLS negotiations, primarily in TLS versions up to 1.2. Each log entry includes details like the certificate's version, serial number, subject, issuer, validity period, key algorithm, and more. This log is crucial for analyzing the certificates used in encrypted communications, helping to identify potentially malicious certificates or unusual certificate chains.

For TLS 1.2 connections, \colorbox{gray!20}{X509.log} entries can be correlated with \colorbox{gray!20}{ssl.log} entries via certificate identifiers, providing a comprehensive view of the encryption parameters and the certificate chain used in a session. This aids in security investigations, such as tracking certificates associated with malicious activity or verifying the legitimacy of encrypted connections.

However, with the introduction of TLS 1.3 and technologies like Encrypted Server Name Indication (ESNI) or Encrypted Client Hello (ECH), the visibility into certificate details is reduced, and in some cases, the \colorbox{gray!20}{X509.log} might not contain entries for TLS 1.3 sessions. This presents challenges for network security monitoring, as the lack of certificate visibility can hinder the ability to analyze encrypted traffic for security purposes.

In summary, \colorbox{gray!20}{X509.log} is a valuable resource for understanding the use of certificates in encrypted network traffic, but evolving encryption standards and practices may limit its utility in some cases.

\\
\vspace{1cm}


\noindent \Large \textbf{smtp.log}
\vspace{5pt}
\\
\normalsize
Zeek's \colorbox{gray!20}{smtp.log} captures and summarizes Simple Mail Transfer Protocol (SMTP) activities, detailing SMTP sessions, including the sending and receiving email addresses, message IDs, subjects, and other relevant data. For unencrypted SMTP traffic on port 25, Zeek can provide a detailed view of the email content and metadata, including attachments if configured to extract files.

However, for encrypted SMTP traffic, typically found on ports 465 (SMTPS) or 587 (submission over TLS), Zeek logs the establishment of the SSL/TLS session in the \colorbox{gray!20}{ssl.log} and can identify the service as SMTP based on the server's name and port numbers. For these encrypted sessions, detailed SMTP transaction information (like sender, recipient, subject) is not available due to encryption, but the session setup, including the chosen encryption cipher and certificate details, can be logged.

Additionally, Zeek does not natively provide detailed logs for other email-related protocols like IMAP (port 143 for unencrypted, 993 for IMAP over TLS) or POP (port 110 for unencrypted, 995 for POP over TLS) beyond recording the SSL/TLS session establishment in \colorbox{gray!20}{ssl.log} for the encrypted variants.

In summary, Zeek's \colorbox{gray!20}{smtp.log} is useful for analyzing unencrypted SMTP traffic, but for encrypted SMTP, IMAP, and POP traffic, the visibility is limited to session setup and encrypted handshake details captured in \colorbox{gray!20}{ssl.log}.
\\
\vspace{1cm}



\noindent \Large \textbf{ssh.log}
\vspace{5pt}
\\
\normalsize
Zeek's \colorbox{gray!20}{ssh.log} captures details about SSH (Secure Shell) sessions, including information about successful and failed login attempts, encryption algorithms used, and the client and server involved in the connection. It can identify lateral movements within a network, outbound connections from an internal network to external servers, and inbound connections from the internet to an internal server. The log includes fields like version, authentication success, authentication attempts, client and server software versions, encryption algorithms, and unique identifiers like the hassh values for profiling SSH client and server configurations. This log is essential for tracking SSH activity, identifying potential unauthorized access, and understanding the encryption standards in use within the network
\\
\vspace{1cm}


\noindent \Large \textbf{pe.log}
\vspace{5pt}
\\
\normalsize
Zeek's \colorbox{gray!20}{pe.log} is designed to capture details specifically related to Portable Executable (PE) files, which are commonly used for Windows executables and libraries. This log records metadata about PE files seen in network traffic, providing insight into potential software and malware distribution over the network. Key information in \colorbox{gray!20}{pe.log} entries includes the compile timestamp, indicating when the executable was built, the machine type (e.g., AMD64 for 64-bit architectures), operating system compatibility, and various flags indicating the use of security features like ASLR (Address Space Layout Randomization) and DEP (Data Execution Prevention).

Additionally, \colorbox{gray!20}{pe.log} details the sections within the PE file, such as .text for executable code and .data for initialized data, which can give clues about the file's structure and purpose. The presence of an import table, export table, debug data, and certificate table in the log entries helps analysts assess the file's interactions with other system components and its authenticity.

In essence, the \colorbox{gray!20}{pe.log} in Zeek is a valuable resource for analyzing the characteristics of Windows executables passing through the network, assisting in identifying suspicious files, understanding software distribution mechanisms, and potentially uncovering malware activity.

\\
\vspace{1cm}



\noindent \Large \textbf{dhcp.log}
\vspace{5pt}
\\
\normalsize
Zeek's \colorbox{gray!20}{dhcp.log} captures the details of DHCP (Dynamic Host Configuration Protocol) communications within a network, facilitating the tracking of IP address assignments and related network configuration details provided by DHCP servers to clients. This log is particularly useful for understanding network dynamics, troubleshooting connectivity issues, and investigating security incidents by mapping IP addresses to MAC addresses, potentially revealing hostnames and identifying DHCP servers active on the network.

The DHCP process involves four key steps, known as DORA: Discover, Offer, Request, and Acknowledge. Zeek's \colorbox{gray!20}{dhcp.log} provides a consolidated view of this exchange, capturing when a client discovers DHCP servers, receives an IP offer, requests an IP, and finally acknowledges the assignment. This log includes important information such as the client's MAC address, requested and assigned IP addresses, hostname, DHCP server address, lease time, and more, all of which are crucial for network administration and security analysis.

By examining \colorbox{gray!20}{dhcp.log}, analysts can identify devices joining the network, track IP address usage over time, and detect anomalies that may indicate unauthorized devices or misconfigurations. This log helps maintain an accurate and up-to-date view of the network's IP address allocation and usage, essential for effective network management and security monitoring.

\\
\vspace{1cm}