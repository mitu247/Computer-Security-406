\subsection{History of Zeek}

Zeek, initially known as Bro, was developed by Vern Paxson in 1995 at the Lawrence Berkeley National Laboratory (LBNL). Designed as a network monitoring tool, it was coined "Bro" to serve as an Orwellian reminder of the surveillance capabilities inherent in network monitoring, with a nod to the potential for privacy violations.
\\\\
Deployed at LBNL in 1996, Zeek's foundational paper won the Best Paper Award at the USENIX Security Symposium in 1998, reflecting its early recognition and significance in the field of network security.
\\\\
With support from the National Science Foundation (NSF) and the Department of Energy (DOE), the project flourished through a blend of academic research and practical application. As Zeek's user base grew, the need for a more user-friendly interface became apparent, leading to a substantial update with the 2.0 release in 2012, in collaboration with the National Center for Supercomputing Applications (NCSA).
\\\\
Subsequent years saw significant enhancements to Zeek, including native IPv6 support, the creation of the Bro Center of Expertise, the launch of try.zeek.org, and the introduction of the Broker communication framework and Zeek package manager.
\\\\
In 2018, to better reflect its community values and avoid the negative connotations of "bro culture," the project was renamed from Bro to Zeek. Version 3.0, released in 2019, was the first to bear the new name. The year 2020 marked a period of renewed focus on community engagement, with increased outreach efforts aimed at expanding the Zeek community.