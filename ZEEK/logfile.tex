\section{Log files in Zeek}
\subsection{conn.log}
The conn.log in Zeek is crucial for tracking network connections, covering both stateful protocols like TCP and stateless ones like UDP. It provides a comprehensive view of network interactions, detailing each connection's aspects. Analysts use this log to understand network behavior, identify anomalies, and investigate incidents. 
\begin{lstlisting}[language=Python, caption= conn.log entry]
{
  "ts": 1591367999.430166,
  "uid": "C5bLoe2Mvxqhawzqqd",
  "id.orig_h": "192.168.4.76",
  "id.orig_p": 46378,
  "id.resp_h": "31.3.245.133",
  "id.resp_p": 80,
  "proto": "tcp",
  "service": "http",
  "duration": 0.25411510467529297,
  "orig_bytes": 77,
  "resp_bytes": 295,
  "conn_state": "SF",
  "missed_bytes": 0,
  "history": "ShADadFf",
  "orig_pkts": 6,
  "orig_ip_bytes": 397,
  "resp_pkts": 4,
  "resp_ip_bytes": 511
}
\end{lstlisting}
A conversation between 192.168.4.76 and 31.3.245.133 occurred on Jun 5, 2020, at 14:39:59 UTC, lasting only 0.254 seconds. They communicated via HTTP on TCP port 80, with 192.168.4.76 sending 77 bytes at the application layer and 397 bytes at the IP layer.

\subsection{dns.log}
Zeek's dns.log captures DNS queries and responses, recording details such as the querying host, the DNS server, the requested domain, and the response. It's used for security and network analysis, helping to identify normal and malicious DNS activities, including domain resolutions and potential misuse of DNS for command-and-control or data exfiltration purposes.
\begin{lstlisting}[language=Python, caption= dns.log entry]
{
  "ts": 1591367999.305988,
  "uid": "CMdzit1AMNsmfAIiQc",
  "id.orig_h": "192.168.4.76",
  "id.orig_p": 36844,
  "id.resp_h": "192.168.4.1",
  "id.resp_p": 53,
  "proto": "udp",
  "trans_id": 19671,
  "rtt": 0.06685185432434082,
  "query": "testmyids.com",
  "qclass": 1,
  "qclass_name": "C_INTERNET",
  "qtype": 1,
  "qtype_name": "A",
  "rcode": 0,
  "rcode_name": "NOERROR",
  "AA": false,
  "TC": false,
  "RD": true,
  "RA": true,
  "Z": 0,
  "answers": [
    "31.3.245.133"
  ],
  "TTLs": [
    3600
  ],
  "rejected": false
}
\end{lstlisting}
According to this log entry, 192.168.4.76 asked 192.168.4.1 for the A record of the host testmyids.com, and received the answer 31.3.245.133. 

\subsection{http.log}
Zeek's http.log captures details of HTTP and exposed HTTPS traffic, recording request and response information like methods, URIs, status codes, and MIME types. It's valuable for analyzing web activities, identifying suspicious or malicious behavior, and correlating with other logs like conn.log for comprehensive network insight.
\begin{lstlisting}[language=Python, caption= http.log entry]
{
  "ts": 1591367999.512593,
  "uid": "C5bLoe2Mvxqhawzqqd",
  "id.orig_h": "192.168.4.76",
  "id.orig_p": 46378,
  "id.resp_h": "31.3.245.133",
  "id.resp_p": 80,
  "trans_depth": 1,
  "method": "GET",
  "host": "testmyids.com",
  "uri": "/",
  "version": "1.1",
  "user_agent": "curl/7.47.0",
  "request_body_len": 0,
  "response_body_len": 39,
  "status_code": 200,
  "status_msg": "OK",
  "tags": [],
  "resp_fuids": [
    "FEEsZS1w0Z0VJIb5x4"
  ],
  "resp_mime_types": [
    "text/plain"
  ]
}
\end{lstlisting}
A request from 192.168.4.76 to 31.3.245.133 was logged by Zeek, showing a HTTP GET request to testmyids.com. The response was "200 OK" with MIME type "text/plain." Zeek assigned a file ID to this exchange, allowing content review if configured. The unique connection ID links this HTTP interaction to the corresponding conn.log entry, facilitating comprehensive network analysis.

\subsection{files.log}
Zeek's files.log records observed files in network traffic, detailing their types and transactions. Analysts can configure Zeek to extract and save specific file types, like executables, for further analysis. This process is tracked through conn.log and http.log to understand file transfer contexts.
\begin{lstlisting}[language=Python, caption= files.log entry]
{
  "ts": 1596820191.969902,
  "fuid": "FBbQxG1GXLXgmWhbk9",
  "uid": "CzoFRWTQ6YIzfFXHk",
  "id.orig_h": "192.168.4.37",
  "id.orig_p": 58264,
  "id.resp_h": "23.195.64.241",
  "id.resp_p": 80,
  "source": "HTTP",
  "depth": 0,
  "analyzers": [
    "EXTRACT",
    "PE"
  ],
  "mime_type": "application/x-dosexec",
  "duration": 0.015498876571655273,
  "is_orig": false,
  "seen_bytes": 179272,
  "total_bytes": 179272,
  "missing_bytes": 0,
  "overflow_bytes": 0,
  "timedout": false,
  "extracted": "HTTP-FBbQxG1GXLXgmWhbk9.exe",
  "extracted_cutoff": false
}
\end{lstlisting}
The files.log entry in Zeek includes a UID from conn.log, linking the file activity to a specific connection. Zeek activated file extraction and analysis, successfully handling 179272 bytes without loss, and saved the file as HTTP-FBbQxG1GXLXgmWhbk9.exe. The is orig field indicates the file's sender; here, the HTTP server at 23.195.64.241 sent the file to 192.168.4.37.
\subsection{ftp.log}
Zeek's ftp.log records FTP communications, detailing client-server interactions including login credentials, commands, and file transfers. It captures both control channel activities and file transfer specifics, highlighting FTP's use of separate TCP connections for commands and data.

\subsection{ssl.log}
Zeek's ssl.log captures metadata for TLS-encrypted sessions, including TLS versions, cipher suites, and elliptic curves. It's critical for understanding the security posture of encrypted communications, offering insights into protocol usage and potential vulnerabilities without decrypting traffic.
\begin{lstlisting}[language=Python, caption= ssl.log entry]
{
  "ts": 1598377391.921726,
  "uid": "CsukF91Bx9mrqdEaH9",
  "id.orig_h": "192.168.4.49",
  "id.orig_p": 56718,
  "id.resp_h": "13.32.202.10",
  "id.resp_p": 443,
  "version": "TLSv12",
  "cipher": "TLS_ECDHE_RSA_WITH_AES_128_GCM_SHA256",
  "curve": "secp256r1",
  "server_name": "www.taosecurity.com",
  "resumed": false,
  "next_protocol": "h2",
  "established": true,
  "cert_chain_fuids": [
    "F2XEvj1CahhdhtfvT4",
    "FZ7ygD3ERPfEVVohG9",
    "F7vklpOKI4yX9wmvh",
    "FAnbnR32nIIr2j9XV"
  ],
  "client_cert_chain_fuids": [],
  "subject": "CN=www.taosecurity.com",
  "issuer": "CN=Amazon,OU=Server CA 1B,O=Amazon,C=US"
}
\end{lstlisting}
This log entry captures a TLS 1.2 handshake between a client and a server, specifying the use of the ECDHE\_RSA key exchange with AES\_128\_GCM\_SHA256 encryption and the secp256r1 elliptic curve. It indicates a connection to "www.taosecurity.com", authenticated with a certificate chain issued by Amazon. The connection upgrades to HTTP/2 ("h2"). No client certificates are used in this session. The log is part of a series, with certificate details to be analyzed in an upcoming x509 log entry.

\subsection{x509.log}
Zeek's x509.log records details from TLS certificates exchanged during secure connections, providing insights into the security of these connections, including the certificate chain and key details. This log is particularly valuable for analyzing certificate validity, issuer authenticity, and encryption standards used in network communications.

\subsection{smtp.log}
Zeek logs SMTP traffic, providing insights into email transactions over various ports, including encrypted ones. It identifies key elements like sender, receiver, and authentication details, even in encrypted sessions, aiding in network security analysis.

\subsection{ssh.log}
Zeek's SSH log provides insights into SSH sessions, detailing aspects like authentication success, encryption types, and client-server interactions, aiding in the analysis of secure remote connections and potential lateral movements within networks.
\subsection{pe.log}
pe.log entry provides details about a portable executable file, including its architecture, operating system compatibility, security features, and structural components. It helps analysts understand the essential characteristics of the executable file quickly.


\subsection{dhcp.log}
dhcp.log provides detailed information about a DHCP (Dynamic Host Configuration Protocol) exchange between a client and a server. It captures the Discover-Offer-Request-Acknowledge (DORA) process, showing the messages exchanged between the client and server, including requests for IP addresses, server responses, assigned IP addresses, lease durations, and other configuration parameters. Additionally, it explains how tools like tcpdump, tshark, and Zeek (formerly known as Bro) can be used to analyze and interpret DHCP exchanges for network and security purposes.

\subsection{weird.log}
weird.log is various random stuff where analyzers ran into trouble understanding the traffic in terms of their protocols; basically whenever there’s something unexpected at the protocol level, that’s a weird (for a lack of anything better to do with it). That means that “weirds” are also essentially hardcoded by whoever wrote that analyzer. They can also be generated by scripts, but that’s rarer.

\subsection{notice.log}
notice.log on the other hand are situations explicitly detected and reported by Zeek scripts as inspection-worthy. It’s usually not protocol errors, but something semantically higher (like a self-signed cert). Notices are part of the script-level analysis and can be raised by Zeek packages as well.




