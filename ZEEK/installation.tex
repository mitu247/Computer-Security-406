
\section{Installation}
Update and upgrade the ubuntu using apt.\\\\
\texttt{sudo apt-get update}\\
\texttt{sudo apt-get upgrade}\\\\
Download the zeek source code from the official website\\
\href{https://zeek.org/get-zeek/}{(https://zeek.org/get-zeek/)}\\
Install dependencies using the below command.\\\\
\texttt{sudo apt-get install cmake make gcc g++ flex bison libpcap-dev libssl-dev python3-dev swig zlib1g-dev}\\\\
Once all the dependencies are installed change the directory to the path where the Zeek source code file is downloaded and unzip the file.\\\\
\texttt{cd Downloads}\\
\texttt{tar -xzf zeek-<verison>.tar.gz}\\\\
Change the directory to the extracted file\\\\
\texttt{cd zeek-<version>}\\\\
Configure zeek using the below command\\\\
\texttt{./configure}\\\\
Once the above command is done run the below commands. Note that this command takes time to execute.\\\\
\texttt{make}\\
\texttt{make install}\\\\
To use zeek as a service we need to add the zeek home directory to the bashrc file.
\\\\
\texttt{nano ~/.bashrc}
\\\\
Add the line below or the home directory file zeek at the end of the file.
\\\\
\texttt{export PATH=/usr/local/zeek/bin:$PATH$}
\\\\
Save and exit the file and to apply changes made run source command and check zeek version and directory.
\\\\
\texttt{source ~/.bashrc}\\
\texttt{which zeek}\\
\texttt{zeek - version}
\\\\
Now change the directory to /usr/local/zeek/etc check the what files are there in the directory.
\\\\
\texttt{cd /usr/local/zeek/etc}\\
\texttt{ls}
\\\\
Open new terminal window and check the ip using the below command check the network interface of the machine.
\\\\
\texttt{ip a}
\\\\
We can see there are 2 interfaces, one is for loopback and other is for broadcast and more which is enp0s1. Note that this may vary from user to user. Note the interface name. Now in the previous window edit node.cfg file using nano and replace the interface name as shown below.
\\\\
\texttt{nano node.cfg}
\\\\
Once the file is saved check if the script is correct, using the below command.
\\\\
\texttt{zeekctl check}
\\\\
Once you get “zeek scripts are ok.” at the end you can deploy zeek, using below command.
\\\\
\texttt{zeekctl deploy}
\\\\
Once zeek is started we can check the status using.
\\\\
\texttt{zeekctl status}
\\\\
Now to view logs we can change the directory to /usr/local/zeek/logs/current
\\\\
\texttt{cd /usr/local/zeek/logs/current}
\\\\
When we use the list command we can see the logs been generated.
\\\\\
We can use tail command to view the logs,
\\\\
\texttt{tail -f conn.log}